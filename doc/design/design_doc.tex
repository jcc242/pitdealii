\documentclass{article}

\usepackage{amsmath}
\usepackage{siunitx}

\begin{document}

\section{Manufactured Solution}
\label{sec:MFGSol}

The method of manufactured solutions is used to verify the implementation of the
parallel in time algorithm.
The method of manufactured solutions specifies a field $u_h$ and an
analytical values for the right hand side, initial conditions, and boundary
conditions is determined.
Once the analytical values are known, they are plugged into the original sovler
and the value of $u$ from the solver is compared to the specified $u_h$ to
determine the error.

For the heat equation,
\begin{align*}
  \frac{\partial u}{\partial t} - \Delta u = f(\boldsymbol{x},t), \qquad &\forall\boldsymbol{x}\in\Omega,t\in\left( 0,T \right), \\
  u(\boldsymbol{x},0) = u_0(\boldsymbol{x}), \qquad &\forall \boldsymbol{x}\in\Omega, \\
  u(\boldsymbol{x},t) = g(\boldsymbol{x},t), \qquad &\forall \boldsymbol{x}\in\partial\Omega,t\in\left( 0,T \right).
\end{align*}
the field $u_h$ is specified as,
\begin{align*}
  u_h = e^{-4\pi^2t}cos(2 \pi x)cos(2 \pi y), \qquad \forall \boldsymbol{x} \in \Omega \cup \partial\Omega
\end{align*}
which gives us the values of,
\begin{align*}
  \frac{\partial u}{\p artial t} &= -4 \pi^2 e^{-4\pi^2t}cos(2 \pi x)cos(2 \pi y), \\
  -\Delta u &= 8 \pi^2 e^{-4\pi^2t}cos(2 \pi x)cos(2 \pi y) \\
  \frac{\partial u}{\partial x} &= -2 \pi e^{-4\pi^2t}sin(2\pi x)cos(2\pi y) \\
  \frac{\partial u}{\partial x} &= -2 \pi e^{-4\pi^2t}cos(2\pi x)sin(2\pi y)
\end{align*}
and therefore we specify the terms from the governing equations as,
\begin{align*}
  f(\boldsymbol{x},t) = 4 \pi^2 e^{-4\pi^2t}\cos(2 \pi x)\cos(2 \pi y), \qquad&\forall\boldsymbol{x}\in\Omega,t\in\left( 0,T \right), \\
  u_0(\boldsymbol{x}) = cos(2 \pi x)\cos(2 \pi y), \qquad &\forall \boldsymbol{x}\in\Omega, \\
  g(\boldsymbol{x},t) = e^{-4\pi^2t}\cos(2 \pi x)\cos(2 \pi y), \qquad &\forall \boldsymbol{x} \in \partial\Omega.
\end{align*}

The convergence rate of the error is calculated with
\begin{align*}
  \Delta \epsilon_n = \frac{\ln{\epsilon_{n-1}/\epsilon_{n}}}{\ln{r_n}}
\end{align*}
where $\Delta \epsilon_n$ is the convergence rate of error $\epsilon$ between a mesh $n$ and
coarser mesh $n-1$ that have a refinement ratio of $r_n$.
Shown in Table~\ref{tbl:convergenceRate} is the convergence rate as the mesh is
refined.
The $\Delta t$ is reduced by a factor of 2 for every global refinement of the mesh.

\begin{table}\label{tbl:errorMFG}
  \begin{center}
    \begin{tabular}{|c|c|c|c|c|c|} \hline
      cycles & \# cells & \# dofs & $L^2$-error & $H^1$-error & $L^\infty$-error \\ \hline
      125 & 48 & 65 & 6.036e-03 & 6.970e-02 & 7.557e-03\\ \hline
      250 & 192 & 225 & 1.735e-03 & 3.414e-02 & 2.721e-03 \\ \hline
      500 & 768 & 833 & 4.513e-04 & 1.690e-02 & 7.410e-04 \\ \hline
      1000 & 3072 & 3201 & 1.140e-04 & 8.426e-03 & 1.877e-04 \\ \hline
      2000 & 12288 & 12545 & 2.859e-05 & 4.209e-03 & 4.715e-05 \\ \hline
    \end{tabular}
  \end{center}
  \caption{The error of the simulation as the mesh is globally refined.}
\end{table}

\begin{table}\label{tbl:convergenceRate}
  \begin{center}
    \begin{tabular}{|c|c|c|c|c|c|} \hline
      cycle & \# cells & \# dofs & Slope $L^2$ & Slope $H^1$  & Slope $\textrm{L}^\infty$ \\ \hline
      125 & 48 & 65 & --- & --- & --- \\ \hline
      250 & 192 & 225 & 1.798 & 1.030 & 1.474 \\ \hline
      500 & 768 & 833 & 1.943 & 1.014 & 1.877 \\ \hline
      1000 & 3072 & 3201 & 1.985 & 1.004 & 1.981 \\ \hline
      2000 & 12288 & 12545 & 1.995 & 1.001  & 1.993 \\ \hline
    \end{tabular}
  \end{center}
  \caption{Convergence rate of the error from Table~\ref{tbl:errorMFG}.}
\end{table}

\section{Serial in Time Performance}
\label{sec:SiTPerformance}

Out of curiosity, I compared the wallclock of the reference serial in time case
against the XBraid implementation run in serial in time.
The linux performance analysis tool \texttt{perf} is used with the following
call,
\begin{verbatim}
perf stat -r 50 ./path/to/binary
\end{verbatim}
which runs the binary 50 times and collects average and deviation information
for task clock, wall clock, cycles, and much more.
The wall clock time is recorded in \ref{tbl:serialPerf} for three different cases.
\begin{table}\label{tbl:serialPerf}
  \centering
  \begin{tabular}{c | c | c | c}
    Number of Cells & Number of Timesteps & Reference & XBraid Serial \\
    \hline
    48 & 2500 & \SI{2.29}{\second} & \SI{3.03}{\second} \\
    48 & 25000 & \SI{21.86}{\second} & \SI{27.85}{\second} \\
    192 & 2500 & \SI{5.46}{\second} & \SI{7.44}{\second} \\
  \end{tabular}
  \caption{Comparison of the run time with various number of cells and number of
  timesteps. There is an overhead cost to running serial in time with XBraid.}
\end{table}

It appears there is an overhead attached to the XBraid serial in time implementation compared
to the reference serial in time.


\end{document}
